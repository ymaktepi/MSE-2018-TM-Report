\chapter{Strategies Evaluation}
\label{chap:strategies}

\section{Modelisation}
\section{Strategies}
The following strategies were proposed in order to visit the entierty of the
trade-off triangle:
\begin{itemize}
        \item randomness;
        \item round-robin;
        \item double round-robin;
        \item overhead.
\end{itemize}
These strategies should allow one to visit the whole triangle and to discuss
their respective strength and weaknesses. The following sections describe the
strategies as well as their expected locations in the triangle.

\subsection{Round-robin}
The first strategy that comes to mind is a simple round-robin. Nodes send
messages one after the other, in a fixed order.

\subsection{Randomness}
The next strategy is the simplest to think of: complete randomness. Using fixed
probability density functions, nodes chose when to create messages and to which
other validator to send them.

\subsection{Double Round-robin}
In this setting, two nodes send messages at the same time, in a fixed order. If
the two nodes that send messages at the same step are at opposite places in the
set of validators \todo{explain better}, the latency to finality is supposedly
half as much as the simple round-robin strategy. The overhead is however
doubled.

\subsection{Maximal Overhead}
This strategy is the most expensive in terms of bandwidth; at each step, each
validator sends a message to the others. This example strategy should give a
baseline value for the maximum overhead that is reachable in the tradeoff
triangle.

\section{Experimentations}
Over the duration of this thesis, the \texttt{core-cbc} library has included a
test framework called \textit{proptest}. The testing framework that has been
implemented includes ways to simulate the behavior of the Casper protocol over
multiple nodes and thousands \todo{numbers} of blocks. At the time of the
writing, the simulations do not include networking latencies.

\todo{schema with what to measure}
\todo{how the measurements take place in the code}

\section{Visualization}
\section{Analysis}

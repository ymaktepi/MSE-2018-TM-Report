\chapter{Testing Framework Implementation}
\label{chap:implementation}
\section{Modelisation}
\subsection{Metrics}
\FloatBarrier
A way to rationally compare strategies is needed in order to discuss their
relative strengths and weaknesses. Three main characteristics will be used for
that:
\begin{itemize}
        \item latency;
        \item number of nodes;
        \item overhead.
\end{itemize}

\subsubsection{Latency}
The latency is the number of messages needed to finalize a block. Ideally, you
want to have a latency as low as possible to reach finality as soon as possible.
The latency is a way to measure liveness in a blockchain system. If it is low,
then the system is considered more "live", as fewer messages are needed in order
to confirm a transaction, and therefore less time.

\subsubsection{Number of nodes}
The number of nodes is quite straightforward; it is the number of nodes that are
in the validator set. This number should be as high as possible to guarantee
decentralization and therefore safety.

\subsubsection{Overhead}
The overhead is the number of messages that are sent over the network between
one step of the consensus and the next. It should be as low as possible to keep
the costs in bandwidth low.

\subsubsection{Example}

\begin{figure}[h]
	\centering
	\includegraphics[width=\columnwidth]{metrics-example}
  \captionsetup{justification=centering}
    \caption{Metrics computation example}
	\label{fig:metricsSchema}
\end{figure}

\fig{fig:metricsSchema} describes how the computation of the metrics
takes place. The left half of the figure shows an initial view of the system. At
time \(t_1\), message \(m_1\) is finalized. The right half of the figure shows a
view of the system at \(t_2\), time at which \(m_2\) is finalized. The latency
of \(m_2\) is \(l_2\), the difference of heights between \(m_3\) and \(m_2\). In
this example, \(l_2 = 4\). \(o_2\), the overhead for \(m_2\), is the number of
messages that have been sent between \(t_1\) and \(t_2\), \(t_1\) being the time
at which the previous message has been finalized and \(t_2\) the time at which
the current message is finalized. In the example, \(o_2 = 3\).
\FloatBarrier

\subsection{Tradeoff triangle/Trilemma}
In a standard consensus protocol, the three metrics form a trade-off triangle in
kind of a "pick two" fashion. \todo{not a pick two} \gls{cbc}-Casper has no
assumptions on timings, sources, contents, destinations of the messages that are
exchanged, and can therefore explore the whole trade-off space. This project
aims to find strategies that span the entierty of the triangle and 
\todo{finish sentence what}

\subsection{Basic model}
\label{ssec:model}
The first model that has been chosen for the evaluation is the following:

\[1 = s_n \cdot \frac{1}{n} + s_l\cdot l + s_o\cdot o\]

This model binds 3 scores \(s_x\) to their respective variables \(x\).
Variables are as follows:
\begin{itemize}
    \item \(n\) the number of nodees;
    \item \(l\) the latency;
    \item \(o\) the overhead.
\end{itemize}
The higher the score \(s_x\) is, the more its related variable is dominant
in the strategy and therefore the closer to a corner of the triangle the strategy is.
\(\frac{1}{n}\) is used because a large number of nodes is wanted.
This model is a basic one, where everything is linear. It is probably not accurate but
will serve as a base for a closer to reality one once data have been plotted using
this model.

\subsection{Model Evaluation}
After running the strategies in the simulation environment, metrics for \(n\),
\(l\) and \(o\) will be recorded. Then, for multiple runs, a linear regression
will be performed in order to find the scores \(s_x\).

\section{Strategies}
\label{sec:strategies}

The following strategies were proposed in order to visit the entirety of the
trade-off triangle:
\begin{itemize}
        \item round-robin;
        \item double round-robin;
        \item triple round-robin;
        \item overhead;
        \item arbitrary.
\end{itemize}
These strategies should allow one to visit the whole triangle and to discuss
their respective strength and weaknesses. The following sections describe the
strategies as well as their expected locations in the triangle.

\begin{figure}[h]
	\centering
	\includegraphics[width=0.8\columnwidth]{all_strats}
  \captionsetup{justification=centering}
    \caption{Strategies examples with 4 validators. Top-left: Round-robin.
    Top-right: Double round-robin. Bottom-left: Maximal overhead. Bottom-right:
    Arbitrary. Arrows here show justifications for all messages.}
	\label{fig:allStrats}
\end{figure}

\subsection{Round-robin}
The first strategy that comes to mind is a simple round-robin. Nodes send
messages one after the other, in a fixed order.


\subsection{Double Round-robin}
In this setting, two nodes send messages at the same time, in a fixed order. If
the two nodes that send messages at the same step are as far as possible from each
other in the set of validators, the latency to finality is supposedly
half as much as the simple round-robin strategy. The overhead is however
doubled.

\subsection{Triple Round-robin}
This strategy is similar to the Double round-robin, except that three nodes send
a message a each step. It was decided to add this strategy at a later stage of
the project in order to have intermediate comparison points between the Double
round-robin and the Maximal overhead strategy. As for the Double round-robin
strategy, the latency should be divided by three compared to the simple
round-robin and the overhead should be three times higher.
Note that a N-uple round-robin could be implemented in order to change the
position in the trade-off triangle but it would lose meaning for a number of
nodes smaller than N so there is no quadruple (or bigger) round-robin.

\subsection{Maximal Overhead}
This strategy is the most expensive in terms of bandwidth; at each step, each
validator sends a message to the others. This example strategy should give a
baseline value for the maximum overhead that is reachable in the tradeoff
triangle. It also minimizes the latency as only two steps are needed in order to
finalize a block. This strategy is similar to the pBFT \todo{cite} protocol, as each
node sends a message to every other one over two steps.

\subsection{Arbitrary}
The next strategy is the simplest to think of: complete randomness. Using fixed
probability density functions, one node is arbitrarily selected at each step to
create a message. 

\subsection{Bottom-up strategies}
\label{ssec:bottomUpStrats}
All the previous strategies (except the Arbitrary one) can be viewed as
"top-down" strategies, as there has to be a way for nodes to be synchronized.
Bottom-up strategies are more likely to be implemented from a real-life point of
view. Each node has its own view of the messages and acts in its own interest
based on that view.
For exemple, a node could keep track of the whole state of the finalization of
messages, and only publish a message itself when it facilitates the
finalization. That kind of strategy has not been implemented but the testing and
scoring framework allows one to easily add them.

Such strategies should implement incentives for the nodes not to spam the
network and include mechanisms to slash nodes not following some rules the
strategies dictate.

\section{Methodology}
Over the duration of this thesis, the \texttt{core-cbc} library has included a
test framework called \texttt{proptest}. The testing framework that has been
implemented includes ways to simulate the behavior of the Casper protocol over
multiple nodes and thousands \todo{numbers} of blocks. At the time of the
writing, the simulations do not include networking latencies.

\subsection{\texttt{proptest}}
The \texttt{proptest} implementation is able to run blockchain simulations off
the following parameters:
\begin{itemize}
    \item Number of validators;
    \item Terminating condition;
    \item Sender strategy;
    \item Receiver strategy.
\end{itemize}

\subsubsection{Number of validators}
This parameter is quite straightforward, it is the number of nodes that can
validate blocks.

\subsubsection{Terminating condition}
The terminating condition is a predicate that tells whether or not the simulation has
reached an end. In our case, the end of the simulation is reached when at least
one node finds a safety oracle for a blockchain that has a height of 4. This
height was chosen because it offered a balance between the computation speed and
the variance in output values. If the chosen value was smaller, the generated
blockchains would all look alike and some side effects would have been observed
near the genesis block. A bigger value would imply more computation time.

\subsubsection{Sender strategy}
The sender strategy selects one or more nodes that will create new messages and
forward them to the rest of the network. All the basic strategies that have been
presented in \sec{sec:strategies} are implemented as Sender strategies.
New strategies (including bottom-up ones) can be easily implemented as well.

\subsubsection{Receiver strategy}
Receiver strategies select a set of validators that receive messages created by
Sender strategies. Two strategies have been implemented for now: 
\begin{itemize}
        \item All receivers;
        \item Some receivers.
\end{itemize}

The \textit{all receivers} strategy broadcasts messages to each other validator.
The \textit{some receivers} strategy sends a message to \(1\) or more validator,
using an uniform probability density function.  As of now, none of the
implemented strategies are a good modelisation of a typical Ethereum network and
this will be fixed at a later iteration.  Nonetheless, these strategies offer
two extreme points on the spectrum of the network topology: a fully connected
one without latency (\textit{all receivers}), and a random one (\textit{some
receivers}) and will be both taken into account to compare the sender
strategies.

\subsection{Metrics measurements}
Metrics are not computed as-is in the \texttt{proptest} implementation. Simpler
data are logged into \texttt{csv} files, and are processed by a \texttt{Python}
script. \todo{what data} This two steps processing permitted the generation of
data before having a precise definition of the metrics. It also permits to vary
the way metrics are recorded. For example, from the same log file, you could
extract metrics only when the last step of the consensus is reached, or have an
average of the metrics at each step of the consensus.



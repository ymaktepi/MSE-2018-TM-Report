\chapter{Background}
\label{chap:background}

\section{\gls{pow} and \gls{pos}}
In Ethereum, \gls{pos} is aiming at replacing \gls{pow} as a distributed
consensus algorithm. This section succintly describe both methods as well as the
main differences between them, and then explains which problems arise when you
replace \gls{pow} with \gls{pos}.

\subsection{What is \gls{pow}?}
\gls{pow} is the current consensus protocol used to decide on a blockchain in
Ethereum. In order To create a new valid block, a node has to solve a
cryptographic puzzle and include its solution in the newly created block. The
difficulty of the puzzle is parametrized in order to have -on average- a block
created at a set interval. A reward is given to the creator of each block.  The
consensus rule states that the chain with the greatest total difficulty is to be
considered the main one. Miners are therefore incentivised to build on the main
chain if they want to get rewards for their work. The fact that the difficulty
changes to keep a certain interval between blocks means that said work is a
proxy for timing; a miner cannot create an arbitrary large number of blocks in a
short time because it's inherent to the protocol.


\subsection{What is \gls{pos}?}
\gls{pos}, on the other hand, selects a new block creator according to its
weight (or stake). This weight can be the node's age, wealth, etc. In this
report, we will mainly discuss a specific \gls{pos} protocol, \gls{cbc}-Casper.

\subsection{\gls{cbc}-Casper}
\gls{cbc}-Casper \todo{cite somewhere} is an abstract consensus protocol family
which is \gls{pos}-ready. Nodes, called validators in this context, send
messages to each other, acknowledging they saw other messages by including them
in a \textit{justification}, that is attached to each message. Based on its
justification as well as a weighted list of validators, each message defines an
\textit{estimate}, which is the consensus value proposed by the sender of the
message. In the case of a blockchain, messages each point to one older message
as their estimate and form a \textit{block-\gls{dag}}. Running a slightly
modified version of the \gls{ghost} algorithm on the \gls{dag} returns a
blockchain.

\todo{change structure; make comparison not between POS and POW but between POW
and CBC}

\subsection{POS vs POW}
\todo{talk about differences between pos and pow }

\begin{table}[H]
    \centering
    {
        \begin{tabular}{|l|p{47mm}|p{47mm}|}
            \hline
            & \gls{pow} & \gls{pos} \\
            \hline
            Block production & Miners can publish blocks if the can prove they
            worked for it & Nodes can publish blocks at any time \\
            \hline
            Timing assumption & Work is a proxy for timing & None \\
            \hline
            Spam & Work removes the possibility to spam & Negligible
            computationnal costs to produce blocks imply potential spam \\
            \hline
            Economic majority & Work is a proxy for economic majority & Economic
            majority \\
            \hline
            Building strategy & Nodes are incentivised to build on the longest
            chain because they have to work for to build blocks & No clear
            incentive to build on the longest chain \\
            \hline
        \end{tabular}
        \captionsetup{justification=centering}
        \caption{Summary of key differences between \gls{pow} and \gls{pos}}
        \label{fig:keyDiffPowPos}
    }
\end{table}
\subsection{key differences}

\subsection{new problems}

\todo{talk about key differences and what this project is about}

\section{CBC casper}
\todo{describe the protocol in my own words}

\todo{describe liveness, safety}
\section{\texttt{core\_cbc}}
\texttt{core\_cbc} a Rust implementation of the \gls{cbc}-Casper, made by
TrueLevel. It implements the consensus algorithms proposed in the paper and
offers an abstract structure that can be used to create consensus on any value.

\section{the other casper?}
\section{Parity Ethereum}
Parity is a Rust Ethereum client. It includes a \textit{Pluggable Consensus}
module that allows one to easily add new consensus protocols by implementing an
interface.  At first, the goal of this project was to implement a small bridge
between the Parity module and the core cbc implementation to test block creation
strategies in a pseudo real-like manner. The implementation of the bridge was
not as straight forward as planned so it has been decided to cut it out and test
strategies without mimicking the network and client settings.

\subsection{Background work}
During the early stages of this project, a clear objective was set: to be able
to run a Casper \textit{testnet} with Parity custom nodes. Some work has been
done for the implementation of the \texttt{core-cbc} library into Parity, but
that was left to people that had a better understanding of the underlying
library. Furthermore, \todo{rephrase "a lot of"} a lot of effort had been
injected in the creation of a Docker infrastructure in order to easily deploy,
connect and monitor multiple custom Parity nodes on a single machine in order to
experiment with different message building strategies. The choice of using
Docker was made because there was a  possibility to work on the UniNE
clusters, which happen to work well with containerized software. After seeing
that the Parity implementation would take too much time to be completed, and
therefore might be unusable for this thesis, it has been decided to evaluate
strategies inside the \texttt{core-cbc} library instead of the more
real-life-like setting that is a Parity testnet. The main disadvantage of doing
the experimentations in the library is that the whole network latencies and
topology are not taken into account. However, a non-negligible advantage of
implementing strategies in the core library is that it will be easier to test
them on consensus values that are not only blockchains.


\chapter{Conclusion}
\label{chap:conclusion}

This project aimed at finding a way to compare block building strategies for a
\gls{cbc} Casper blockchain protocol. Based on a core library implementing the
abstract structure for a \gls{cbc} Casper blockchain, a testing framework has
been built. The framework is generic enough to allow one to implement new block
building strategies as well as new network message passing topologies and
improved termination condition. Three main variables have been selected to
compare strategies: latency, number of nodes and overhead. The framework has
been validated using basic strategies and the explicit relative performances
they are expected to show. By slightly adapting the model to the expected
performances of the strategies, a new model has been proposed. The new model
shows a better comparison mean than the first one and but is still weak to
estimate the scaling of the number of nodes of the different strategies, even
though its importance is lower than the simpler model.

The main improvements that are still to be added are a better modelisation with
respect to the number of nodes variable, a network topology
implementation that reflects a real-life setting, as well as ``bottom-up''
strategies, that are more likely to be implemented in a real blockchain than the
basic ones used to validate the framework. Furthermore, data that has been
generated with other sending strategies have not been analyzed yet and could be
used to optimize the model.
